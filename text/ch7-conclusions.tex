\chapter{Conclusions}\label{ch:7-conclusions}
The MicroBooNE experiment was designed to search and to clarify the nature of the low-energy excess of electron-like events observed by the MiniBooNE experiment. 

We implemented an algorithm which employs the information coming from the optical system and the TPC of the MicroBooNE detector to select a sample enriched with $\nu_e$~CC0$\pi$-Np events. We showed that it is possible to reject the cosmic and neutrino backgrounds by applying rectangular cuts on kinematic and calorimetric variables or by exploiting the classification power of the Boosted Decision Trees. In particular, we showed that the energy loss per length $dE/dx$ can be used to distinguish between electrons and single photons interacting in a LArTPC. 

The selection was applied on a small sub-sample of the data collected by triggering on the Booster Neutrino Beam, corresponding to $4.34\times10^{19}$~POT. In our Monte Carlo simulation we selected $3.2\pm0.8$ ($4.4\pm1.2$) $\nu_e$~CC0$\pi$-Np events and $12.0\pm3.0$ ($5.5\pm1.5$) background events after the application of the rectangular cuts (BDTs cuts), in good agreement with the $16$ ($11$) selected data events. The validation was performed on two orthogonal side-bands enriched with NC and CC $\nu_{\mu}$ interactions. The systematic uncertainties related to the cross-section, flux, and detector simulation were evaluated. The selection was also applied to an independent data sample acquired by triggering on the NuMI beam. The significance of the presence of electron neutrinos in the NuMI beam was calculated, giving solid confidence in the analysis presented. 

We measured the current sensitivity to the low-energy excess with an expected exposure of the MicroBooNE detector of $13.2\times10^{20}$ POT. The performances needed to reach a $5\sigma$ sensitivity were estimated.

The analysis showed here will serve as the foundation of the final low-energy excess search in the electron hypothesis. We have developed the full analysis framework, including the assessment of the systematic uncertainties, which can now be used with the improved simulations to measure the definitive MicroBooNE sensitivity to the low-energy excess. 

\begin{figure}[htbp]
    \centering
    \includegraphics[width=0.98\linewidth]{figures/sbn_aerial.pdf}
    \caption{Aerial view of the Fermilab campus with the position of the three detectors forming the future Short Baseline Neutrino program. The neutrino beam target is placed on the right side of the picture. From \cite{Antonello:2015lea}.}
    \label{fig:sbn_aerial}
\end{figure}

Regarding the future, MicroBooNE is the first stage of the broader Short Baseline Neutrino (SBN) program at Fermilab \cite{Antonello:2015lea}. In this program, two other LArTPCs will be placed on-axis with the Booster Neutrino Beamline. The Short Baseline Neutrino Detector (SBND, formerly known as LAr1-ND \cite{Adams:2013uaa}) will be placed close to the neutrino beam target and the refurbished ICARUS T600 detector \cite{Varanini:2017pvw} will be placed 600~m far from the target, 130~m farther than MicroBooNE (Figure \ref{fig:sbn_aerial}).

\begin{figure}[htbp]
    \centering
    \includegraphics[width=0.75\linewidth]{figures/sbn_sensitivity.pdf}
    \caption{Sensitivity of the SBN Program to $\nu_{\mu}\rightarrow\nu_e$ oscillation signals. The filled areas correspond to the current allowed regions and the lines represent the sensitivity of the SBN program at different confidence levels. From \cite{Antonello:2015lea}.}
    \label{fig:sbn_sensitivity}
\end{figure}

The presence of three detectors on the same beamline employing the same technology will allow to constrain the flux, cross-section, and detector systematic uncertainties. The data coming from three detectors placed at different distances will allow to observe an eventual oscillation pattern and set the best limit on the search for sterile neutrino oscillations in the eV region. Figure \ref{fig:sbn_sensitivity} shows the sensitivity of the SBN program to $\nu_{\mu}\rightarrow\nu_e$ oscillations, assuming background rejection through topological cuts and with the aid of an external cosmic-ray tagger. 

The next-generation large-scale neutrino experiment will be DUNE, which aims to precision measurement of the PMNS mixing angles and to identify the neutrino mass hierarchy. The ultimate goal is to verify the presence of CP violation in the leptonic sector \cite{Acciarri:2015uup}. The design of DUNE employs 40~kton LArTPCs and the know-how with this technology acquired by MicroBooNE will be of fundamental importance for the success of the experiment.
Addressing the presence of sterile neutrinos is also of the utmost importance for DUNE, as their existence would affect the interpretation of the oscillation patterns observed \cite{Gandhi:2015xza}.