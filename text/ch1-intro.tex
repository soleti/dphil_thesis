\begin{savequote}[8cm]
 Secondo la proposta di Pauli si può, ad esempio, ammettere l'esistenza di una nuova particella, il così detto neutrino avente carica elettrica nulla e massa dell'ordine di grandezza di quella dell'elettrone o minore.
 \qauthor{--- Enrico Fermi \cite{Fermi:1933jpa}}
\end{savequote}

\chapter{\label{ch:1-intro}Introduction} 

This thesis describes the first fully-automated electron neutrino search in a Liquid Argon Time Projection Chamber (LArTPC) and the work towards the search for a low-energy excess of electron neutrinos at MicroBooNE.

The history of neutrinos begins with an anomaly: the continuous energy spectrum of the nuclear beta decay could not be explained with a two-body decay. It took more than 25 years to experimentally confirm their existence, but neutrinos continued to puzzle experimentalists until the early 2000, when the existence of neutrino oscillations was finally settled. 
However, in the last two decades, several experiments collected results not fully agreement with a 3-generation scenario.
In particular, LSND first, and MiniBooNE later, found an excess of electron neutrinos compatible with the existence of a fourth, non weakly-interacting, neutrino. The goal of the MicroBooNE experiment is to definitely clarify the nature of the excess.

In \textbf{Chapter \ref{ch:2-neutrinophysics}} we will provide a brief theoretical introduction to the theory of neutrino oscillations and to the main experimental techniques employed to detected them.
In \textbf{Chapter \ref{ch:3-anomalies}}, the LSND and MiniBooNE experiment will be described. A brief overview of other anomalous results and their possible theoretical interpretations will also be provided.
The MicroBooNE experiment will be described in \textbf{Chapter \ref{ch:4-microboone}}. We will enumerate the physics goals of the experiment and explain the main features of the detector, with an overview of the artificial neutrino beams at Fermilab.
The techniques employed to reconstruct the signals coming from the detector will be described in \textbf{Chapter \ref{sec:eventreco}}. The pattern recognition is performed by the Pandora framework, which will be briefly outlined.
Finally, \textbf{Chapter \ref{ch:6-analysis}} will thoroughly describe the fully automated electron-neutrino selection and its validation. The performances required to assess the low-energy excess will also be evaluated.
The thesis ends with a summary of the results and an overview of the future prospects for MicroBooNE and for the sterile neutrino searches in general. 

\vspace{1em}

\textbf{Appendix \ref{sec:mucs}} contains a publication, whose corresponding author is the author of this thesis, with the first measurement of the cosmic-ray reconstruction efficiency in a LArTPC.