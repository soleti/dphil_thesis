\begin{savequote}[8cm]
 Secondo la proposta di Pauli si può, ad esempio, ammettere l'esistenza di una nuova particella, il così detto neutrino avente carica elettrica nulla e massa dell'ordine di grandezza di quella dell'elettrone o minore.
 \qauthor{--- Enrico Fermi \cite{Fermi:1933jpa}}
\end{savequote}

\chapter{\label{ch:1-intro}Introduction} 

This thesis describes the first fully-automated electron neutrino search in MicroBooNE and the work towards the search for a low-energy excess of electron neutrinos.

The history of neutrinos begins with an anomaly: the continuous energy spectrum of the nuclear beta decay could not be explained with the presence of only two particles in the final state. It took more than 25 years to experimentally confirm their existence, but neutrinos continued to puzzle experimentalists and theorists until the early 2000s, when the existence of neutrino oscillation was finally settled. 
Furthermore, in the last two decades, several experiments collected results not fully in agreement with a 3-neutrino scenario.
In particular, the Liquid Scintillator Neutrino Detector (LSND) first, and MiniBooNE later, found an excess of electron-like events, which could be explained with the existence of a fourth, non-weakly-interacting, neutrino. However, the MiniBooNE experiment is a Cherenkov detector and is not able to distinguish between single photons and electrons in the final state. The LArTPC technology, instead, offers excellent granularity and powerful separation between electrons and photons. The goal of the MicroBooNE experiment is to search for, and definitely clarify the observation of this electron-like low-energy excess.

In \textbf{Chapter \ref{ch:2-neutrinophysics}} we will provide a brief theoretical introduction to neutrino oscillations and to the main experimental techniques employed to observe them.
In \textbf{Chapter \ref{ch:3-anomalies}}, the LSND and MiniBooNE experiments will be described. A brief overview of other anomalous results and their possible theoretical interpretations will also be provided.
The MicroBooNE experiment will be described in \textbf{Chapter \ref{ch:4-microboone}}. We will enumerate the main physics goals of the experiment and explain the main features of the detector, with an overview of the Booster Neutrino Beam at Fermilab.
The techniques employed to reconstruct the signals coming from the detector will be described in \textbf{Chapter \ref{sec:eventreco}}. The pattern recognition is performed by the Pandora framework, which will be briefly outlined.
\textbf{Chapter \ref{ch:6-analysis}} will thoroughly characterise the fully automated electron-neutrino selection, which aims to obtain a sample enriched with $\nu_e$~CC0$\pi$-Np interactions. Two background-rejection techniques will be used: one with rectangular cuts on kinematic and calorimetric variables and one employing Boosted Decision Trees. The energy reconstruction and the measurement of the energy loss per length, essential for electron/photon separation, will be outlined. The validation of the selection will be performed with a study on orthogonal side-bands enriched with neutral-current and charged-current $\nu_{\mu}$ interactions and by applying the selection on an independent data sample, containing neutrino interactions from the NuMI beam. The systematic uncertainties in the selection, caused by cross section, flux, and detector effects, will be estimated in \textbf{Chapter \ref{sec:systematics}}.
In \textbf{Chapter \ref{sec:sensitivity}} the sensitivity to the low-energy excess of the MiniBooNE experiment in the electron hypothesis will be calculated. The performances required to reach a $5\sigma$ sensitivity will also be evaluated.
The thesis ends in \textbf{Chapter \ref{ch:7-conclusions}} with a summary of the results and an overview of the future prospects for MicroBooNE and for neutrino physics in general. 

\vspace{1em}

\textbf{Appendix \ref{sec:mucs}} contains a publication, whose corresponding author is the author of this thesis, with the first measurement of the cosmic-ray reconstruction efficiency in a LArTPC.