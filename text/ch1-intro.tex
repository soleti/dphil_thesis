\begin{savequote}[8cm]
 Secondo la proposta di Pauli si può, ad esempio, ammettere l'esistenza di una nuova particella, il così detto neutrino avente carica elettrica nulla e massa dell'ordine di grandezza di quella dell'elettrone o minore.
 \qauthor{--- Enrico Fermi \cite{Fermi:1933jpa}}
\end{savequote}

\chapter{\label{ch:1-intro}Introduction} 


The rapid advance of minimally-invasive cardiac procedures promises improvements in patient safety, procedure efficacy, and access to treatment.  While percutaneous coronary intervention (PCI) has become routine and highly effective , catheter procedures in areas such as electrophysiology (EP) and valve replacement are still coming of age.  This progress is driven by demographics and the improvement in general cardiac care, as patients surviving initial cardiac events go on to require treatment for sequelae .  The growing need for advanced treatment is being answered by developments in catheter technology and procedures.  These tools are continually advancing to access and manipulate an ever-broader range of anatomy.
