Three (and a half) years looked like a long time when I started my DPhil at Oxford. However, thank to a large number of people, this period passed in the blink of an eye: time flies when you're having fun!

First of all I would like to thank my advisor Roxanne Guenette. You have been not only an excellent mentor but also an extremely supportive, helpful and friendly person. I am very grateful for the countless quick chats we had about physics, work, and life. Few lines are not enough to thank you for the opportunities you gave me during the DPhil and the once-in-a-lifetime chance to work with you at Harvard. I have been very fortunate.

I also had the luck to work with great colleagues, who gave me advice throughout these years and helped me to produce the results shown in this thesis. Thank you Marco, Wouter, and Nicolò. A special mention goes of course to Corey, Matt, and Justo, the best postdocs a DPhil student can ask for.

Being part of MicroBooNE, I also had the opportunity to work with some of the most talented physicists in the world. The special environment offered by Fermilab allowed me to grow as a physicist and as a person.

My time in Oxford wouldn't have been the same without the amazing friends I made here. Thanks to Andrea, Giovanna, Angela, Mattia, Virginia and many others for the innumerable dinners at CS.
Aslo my period in Boston felt like home, thanks to Cinzia, Carmela, Giorgio, Steven and the countless number of "Italians" I met there. 

It is not common to keep friends in your hometown when you are there a couple of weeks every year. However, Vito, Valerio, Mattia, Leonardo, and Francesco were always ready to go out whenever I was there.
Even if far, my family was always very supportive of all my choices and ready to give me a warm welcome back every time I was at home. I wouldn't be here if it wasn't for them.