The Micro Booster Neutrino Experiment (MicroBooNE) is a Liquid Argon Time Projection Chamber (LArTPC) for short-baseline neutrino physics at the Fermi National Accelerator Laboratory. The main physics goal of MicroBooNE is to clarify the nature of the low-energy excess of electron-like events observed by the MiniBooNE experiment.
The MiniBooNE experiment is a Cherenkov detector and this technology does not allow to distinguish between electrons and photons in the final state. 
LArTPC detectors, instead, offer excellent granularity and powerful electron/photon separation. For this reason, they represent an ideal technology for the detection of electron neutrino interactions.
This thesis presents the first fully-automated electron neutrino selection in a LArTPC. The selection looks for charged-current electron neutrino interactions with zero pions and at least one proton in the final state. It is applied on a sub-sample of the data acquired by the detector triggering on the Booster Neutrino Beam, corresponding to $4.34\times10^{19}$~protons-on-target. The validation is performed on two orthogonal side-bands, enriched with neutral-current and charged-current muon neutrinos interactions, respectively. The effect of the cross-section, flux, and detector simulation uncertainties is also evaluated. 
The MicroBooNE detector is placed off-axis with the Neutrinos at the Main Injector (NuMI) beam. An independent dataset of events acquired by triggering on the NuMI beam is used to measure the significance of the detection of electron neutrinos in the beam. 
The sensitivity of the MicroBooNE experiment to the MiniBooNE low-energy excess in the electron hypothesis is evaluated. The efficiency and background-rejection power necessary to achieve $5\sigma$ sensitivity are quantified.