% Validation section should include: 
% Data/MC Agreement after precuts on all important variables
% Sideband Checks
% Corsika In-time vs BNB-ext
% Future Validation Studies

\section{Validation}

\subsection{Side-bands checks}
In this section we will study the agreement between data and Monte Carlo for selected samples orthogonal to our $\nu_{e}$ CC0$\pi$-Np signal. In order to validate our analysis, some of the background cuts described in Section \ref{sec:bkg} are inverted or removed in order to enhance different background components.
\subsubsection{Photon-enhanced selection}
It is possible to enhance the neutral-current component (defined as \emph{beam intrinsic NC} in our analysis) by (1) inverting the cut on the shower $dE/dx$, , and (2) removing the cut on the shower distance (see Figures \ref{fig:dedx_norm}, \ref{fig:showerd_norm}). The $dE/dx$ of the most energetic shower must be within 3.2~MeV/cm and 5~MeV/cm to select electromagnetic cascades that were initiated by a photon. It also ensures that this photon-enhanced sample is orthogonal to the $\nu_{e}$ CC0$\pi$-Np selected sample. The cut on the shower distance is removed to include events where the photon conversion is far from the neutrino interaction vertex.
Thus, our final sample will mainly contain NC events, with some contamination of $\nu_{\mu}$ CC$\pi^{0}$ events where the muon track was tagged as a proton-like track.

Figure \ref{fig:photon} shows the comparison between data and Monte Carlo for the reconstructed energy spectrum $E_{deposited}$ of the photon-enhanced event spectrum. 
%The reconstructed energy $E_{corr}$ here corresponds to the sum of the reconstructed energies of the shower-like objects and the reconstructed energies of the track-like objects $E_{corr} = E_{corr}^{p}+E_{corr}^{e}$.

\begin{figure}[htbp]
\centering
  \includegraphics[width=0.7\linewidth]{figures/nc_reco.pdf}
  \caption{Reconstructed energy spectrum of the events selected with the photon-enhanced reverse cuts.}\label{fig:photon}
\end{figure}

\subsubsection{CC \texorpdfstring{$\nu_{\mu}$}{numu}-enhanced reverse cuts}
It is possible to enhance the presence of the CC $\nu_{\mu}$ background (defined as \emph{beam intrinsic $\nu_{\mu}$} in our analysis) by (1) removing the cut on the total number of hits in the collection plane, (2) removing the cut on the fraction of shower hits, (3) requiring a minimum track length, (4) requiring at least a track with $40 < \chi^{2} < 220$ (muon-like track), and (5) requiring that the event is selected by the external $\nu_{\mu}$ CC-inclusive analysis \cite{ubxsec} (see Figures \ref{fig:length_norm}, \ref{fig:proton_norm}). Also in this case the CC $\nu_{\mu}$-enhanced sample will be orthogonal to the $\nu_{e}$ CC0$\pi$-Np selected sample.
A CC $\nu_{\mu}$ event has, by definition, a muon in the final state: as such, requiring a track length larger than 20~cm and changing the cut on the proton $\chi^2$ decreases our muon-rejection power. The goal of the external analysis is to select CC $\nu_{\mu}$ events, so instead of vetoing those events as described in Section \ref{sec:numu}, we invert this requirement by allowing only these events.

Figure \ref{fig:numu_inverted} shows the agreement between data and Monte Carlo for the reconstructed energy spectrum of the CC $\nu_{\mu}$-enhanced event spectrum.

\begin{figure}[htbp]
\centering
  \includegraphics[width=0.7\linewidth]{figures/numu_reco.pdf}
  \caption{Reconstructed energy spectrum of the events selected with the CC $\nu_{\mu}$-enhanced reverse cuts.}\label{fig:numu_inverted}
\end{figure}

\subsection{Future Validation Studies}

\subsubsection{Cosmic-ray studies}
In order to validate the cosmic-ray components of our selected events it is possible to compare simulated events with a CORSIKA cosmic ray producing a flash in the optical system during the beam-gate window and the data off-beam sample. 
In this way we will be able to check if the distributions of the variables we use (e.g. shower energy, shower $dE/dx$) show a good agreement between the simulation and a well-understood set of data events. 
It will help to validate the cosmic background components and also the energy and $dE/dx$ reconstruction procedures.

\subsubsection{NuMI beam event studies}
It is possible to run this analysis on the complementary NuMI dataset. The NuMI beam is created from 120 GeV protons hitting a carbon target, while the BNB is created from 8 GeV protons on a beryllium target. NuMI has also a higher beam intrinsic $\nu_{e}$ component than BNB (5\% vs. 0.5\%). Even though it is off-axis, MicroBooNE will still receive $\sim2500$ $\nu_{e}$ interactions per year. 
As such, a study of the events selected in the NuMI dataset is of fundamental importance to validate the $\nu_{e}$ CC0$\pi$-Np selection algorithm.
%in a different energy region, where the effect of the MiniBooNE low-energy excess should be negligible.

% \begin{figure}[htbp]
% \centering
%   \begin{subfigure}{0.45\textwidth}
%     \includegraphics[width=\linewidth]{figures/numi.pdf}
%     \caption{NuMI beam flux.} 
%   \end{subfigure}
%     \begin{subfigure}{0.45\textwidth}
%     \includegraphics[width=\linewidth]{figures/bnb.pdf}
%     \caption{BNB beam flux.} 
%   \end{subfigure}
%   \caption{NuMI and BNB neutrino fluxes for each neutrino and antineutrino component, when the beams are in neutrino mode.}\label{fig:numibeam}
% \end{figure}

